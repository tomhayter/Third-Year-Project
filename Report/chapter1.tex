\chapter{Introduction}

\section{Overview}

My project, \textit{``An ontology-based restaurant information system''}, explores how ontologies can be used to store information about a restaurant's menu. In a world where over half of the UK population are eating out at restaurants and fast food chains multiple times a month \cite{wunsch_2022}, and a far greater emphasis is being placed on dietary requirements, it is vital that food vendors are able to provide such information to customers. While some customers have specific diets such as vegetarianism and veganism, others have specific allergies or are simply looking to reduce their calorific intake. Religious diets also play a huge part in what we can and cannot eat. Therefore, the restaurant staff want to be able to provide sufficient information to customers about what dishes are available to them based on their requirements. There is also the aspect of modifying dishes. For example, \textit{``Is a plant burger with extra bacon still suitable for vegetarians?''} and \textit{``How many calories are in a cheese burger without the cheese?''}.

In this project I will explore to what extent it is possible to store the necessary information from a menu in an ontology. For example, whether you can store calorific information about ingredients, calculate dishes suitable for certain diets and calculate new information about a dish with certain ingredients removed or extra ingredients added.  I will also compare different ways of storing the same information and find the method that is both simple and effective. I will then evaluate whether an ontology can be used as a way to store information in a system that is used by those who are not familiar with ontologies and how they work.

The final product will consist of two user interfaces, the ontology itself, and the code to link the two together. The first user interface (UI) will be targeted at the customer of the restaurant. They will need to view and filter the menu based on their dietary requirements. They will also be able to modify dishes by adding or removing ingredients. Then, they will be able to see the updated information for the modified dish. The second will be aimed at the owner of the restaurant. They need to be able to create and edit the menu for the customers.

\section{This Report}

This report is broken down into five sections. In the remaining part of the introduction chapter I will describe the planning of the project. The second chapter focuses on the background to the project, including information about diets, technologies that I have used and how ontologies work.

The third chapter discusses all of my design choices for the project and evaluation. It details the methodology used to create the final system, and also lists my requirements and success criteria for the project. The fourth chapter is where I describe the final product and describe the results of the user studies that I conducted. The final chapter evaluates the project as a whole, discusses whether I met my objectives, what I could have done better and explores where this project could be developed further.

\section{Project Plan}

The project ran from September 2022 until April 2023. I broke the project down into multiple stages: research, development and testing. I would spend the first few months researching how ontologies work and following tutorials to understand how different technologies works. This would include getting some hands on experience by using the technologies myself to make sure I truly understood how they worked. Then I would design the system that I wanted to create, and develop a minimal viable product that fits this. Then I would pad the system out with all the required features. The next step was to run a pilot test, to recieve some early feedback about the system. Then, in the final weeks of development, I will finish the system and make changes based on the feedback from the pilot test. Once it is finished, I will conduct another user study to evaluate the success of the system.


% Local Variables:
% mode: latex
% TeX-master: "report"
% End:
