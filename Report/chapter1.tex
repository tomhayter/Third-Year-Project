\chapter{Introduction}

\section{Requirements of the ontology}

These are the requirements of the ontology so that it can store all the information that I would like it to:

\begin{itemize}
\item Ingredient list for every meal on a menu, split into the components of the dish.
\item Calorific content of each meal.
\item Allergen information for each meal.
\item Types of customer e.g. Vegetarians and those with Coelaic disease.
\end{itemize}

These are possible additions that could be made to the ontology, but are not necessary for the MVP:

\begin{itemize}
\item The ability to query a dish with ingredients added or removed.
\item The ability to query a dish based on how a dish has been cooked, e.g. Which meals have not used a deep fryer to be made?
\end{itemize}

The requirements of the user interface for restaurant owner:

\begin{itemize}
\item Add and remove meals using ingredients and components list.
\item Add and remove ingredients and components, in same page as above and separate.
\item Add and remove customer types (dietry requirements and allergens).
\item Query the ontology.
\end{itemize}

Dietary Requirements to filter by:

\begin{itemize}
\item Vegetarian
\item Pescetarian
\item Vegan
\item
\end{itemize}

Allergies\cite{burks2001food}:

\begin{itemize}
\item Peanut
\item Ceoliac (gluten)
\item Wheat
\item Cow's milk
\item Eggs
\item Fish
\item Shellfish
\item Tree nuts
\item Soybeans
\end{itemize}

\section{Evaluation Plan}

To test the ontology system, I plan to conduct user studies, where I will ask people that I know to carry out certain tasks using the system, such as adding or removing a dish from the system, and searching for meals with certain parameters. These tasks will cover each aspect of the functional requirements of the system to completely evaluate how successfully the system meets the criteria.

It is important that the participants of the study are comfortable and give an honest review of the system. To do this, I aim to avoid putting them under any pressures. I will not place them under any time constraints to complete the tasks, and make sure that when I propose the study to them that the estimation is accurate, while also a slight overestimate. It is also important that the participant doesn't feel any pressure to falsely support the system, when it is in fact failing.

% Everything below here is commented material which is used by the
% emacs tex support system called auctex. If you're not an emacs user
% you can safely ignore it. If you do use emacs you should take a look
% at the local emacs or LaTeX WWW pages for more on emacs support for
% LaTeX.

% Local Variables:
% mode: latex
% TeX-master: "report"
% End:
