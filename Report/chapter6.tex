\chapter{Summary \& Outlook}
\label{section:Evaluation}

\section{Did I Meet My Objectives?}

Overall, I believe I met all the necessary objectives for this project. Looking back to the System Requirements (see Section \ref{section:SystemRequirements}), I created a system that was able to store all the dishes on a restaurant's menu, along with the ingredients in the dish. The system could list the dishes to a customer, and they could filter the dishes by various dietary requirements. The system also had the to option to hide calorie information. The owner of a restaurant could manage the menu by adding, removing and editing ingredients and dishes. I also managed to include the ability to generate new information after a customer modified the ingredients in the dish. I also managed to include filters based on how a dish had been prepared. I gave the restaurant owner the ability to add new allergens to the system, however one place that the system falls short is that they can't add new diets easily.

In terms of the non-functional requirements, participants believed that the system was intuitive, easy to use, and that it provided appropriate feedback. They also stated that they felt in control of the system at all times, and could easily reverse any actions they made. The system was reported as being consistent, and did not require the user to remember lots of information. Therefore, I believe it adheres to \textit{The eight golden rules of interface design}. On top of this, participants were able to use the system with either no training if they acted as a customer, or minimal training if they acted as a restaurant owner. Finally, all participants found that the user interfaces ran fast.

In terms of dietary requirements, the system caters for all required allergens, the vegetarian and vegan diets, and the halal and kosher diets. It also provides information regarding whether gluten free alternatives are available for dishes. However, it could be improved. Sikh's do not eat halal or kosher foods \cite{guidance_on_foods_for_religious_faiths_2009}, as it is forbidden for them to eat meat from animals slaughtered according to other religious guidelines. Currently in the system, you cannot choose to remove Halal or Kosher options, only to remove the dishes that are not Halal or Kosher.

\section{Improvements}

There are still many improvements that can be made to the system. First of all, the appearance of the interfaces could definitely be improved. The aesthetics were not the focus of this project and I solely focused on the functionality, so there is great room for improvement. The visuals were mentioned during Study 2 by one user of the RestaurantUI, who mentioned that it was difficult to differentiate between all the different menu pages without reading the title as they are all very similar. Improving the visuals and making it easier to differentiate between the different pages would help this, as long as the overall appearance still remained consistent.

Another addition that could be made is the inclusion of a text search at the top of lists. The list of ingredients and components can grow very long, even with the ingredient type filter, and it can be easy to miss things - whereas a text search would make it easy to find specific items. This was also mentioned during Study 2.

On top of this, a lot of the text handling could be improved. At the minute, no class can have a space in the name of it as the system can't handle it. Spaces are stored differently in the ontology file and not handled when reading from it. The solution I would implement now is the following: When receiving an input of text from the user interface, remove any spaces and convert the text to pascal case before passing it to the OWL API. Then, whenever the name of an entity is received from the ontology to be displayed: enter a space before every capital letter except the first letter. This would work because the standard for storing things in an ontology is in pascal case, but the names of entities would displayed with spaces in the user interfaces.

Another text handling issue is that you can't include the words ``Dish'', ``Component'', ``Ingredient'' etc. within the name of a class. This is because they are used as suffixes in the ontology, and all instances of the suffix is removed when displaying names from the ontology. This would be a simple fix, by making sure only the final occurrence of that text is removed from the name of a class, so that only the actual suffix is removed.

One improvement that was brought to light during the final user study that is specific to the Customer UI regards the settings page. Two of the three users stated that since you spend so much time on the search page, you forget about the main menu page. Therefore, it is easy to forget about the settings page, which can only be navigated to from the main menu page. They suggested being able to navigate to it from the search page instead, which would reduce the cognitive load of the user. This would also mean the Customer UI could be simplified further, and the home page could be removed entirely, making the UI even more lightweight and easier to navigate.

One issue I had throughout this project was what to call the collective group of specific foods that a customer could filter out. While a lot of these foods are allergens, I also included beef in this list to account for those who do not want to eat beef or beef products for reasons other than allergies. I have called them allergens in the user interface, but nutrients in the ontology. In my opinion, the term ``Allergen'' conveyed best the functionality that is inhibited by this list of foods in that a customer would want to view dishes without these foods. However, I used the term `Nutrient' in the ontology because I believed that term encompasses the list of foods most accurately. It would improve the clarity of the system to have one term consistently throughout. 

Another consideration I had at the end was the purpose of the Component. With the particular menu I have chosen, components are not particularly necessary. A burger is a whole meal and it does not consist of definitive components, just ingredients. This caused some confusion at what components to make during the studies when the user was given free reign to make a dish. Other menus lend themselves more to the idea of a Component. If your dish is a chicken and mushroom pie with mashed potato and roasted vegetables - you have three very clear components to the final dish made up of many ingredients themselves. The component architecture lends itself to larger dishes that contain multiple parts, rather than menus where you order lots of small bits. The reason I decided to keep components in is because it gives the system flexibility to be used in different kinds of restaurants. If components are not needed, you could still make the system work by making every dish a Component, and then the Dish in the system would only contain one Component.


\section{Conclusion}

Primarily, the results of Study 2 (see Section \ref{section:Study2Results}) reflect well upon the system. All participants were able to complete all the tasks they were set, which cover the functional requirements of the system (see Section \ref{section:FunctionalRequirements}). They also reported that the system was easy to use, fast to respond and intuitive.

Therefore, I conclude that ontologies can be used to represent the necessary ingredient and dietary information stored on a menu. You can store information about many different types of dietary requirements, and even if they were not included in my final system - I discovered there is the ability to store them in an ontology. Importantly, it is possible to create a user interface that hides all the complex logic of an ontology. The user interfaces allow those who have no knowledge of ontologies to use a system that stores information in them without needing to know how they work. In this context, the only knowledge required is the hierarchical nature of ingredients, components and dishes.

Compared to a traditional database, I believe that an ontology makes it easier to represent real world objects and relationships, which makes it more applicable in a context like this. Ontologies also allow for information to be calculated, such as whether dishes are suitable for vegetarians and what allergens are present in a dish. In a traditional database system, this information would need to be input by a user and stored in a field that contained a boolean value. A database system would also require the user to enter a list of allergens for each dish, again increasing the workload for the owner of a restaurant. While it could take longer to set up, I believe the ontology-based system would save the owner of a restaurant time in the long run.

I believe this system, with some adjustments, could be used in a restaurant to manage and display dietary information. With some adjustments discussed in Section \ref{section:FutureWork}, I believe this system could be transformed into a fully functional electronic menu system. I also believe that ontologies can be used in real world applications where the size of the ontology is not going to be large. They have an intuitive nature about them and make it easy to represent real world objects.

\section{Future Work}
\label{section:FutureWork}

There are many possibilities when it comes to extending this project. The first is to add a transactional side to it, so that a restaurant owner would be able to add prices and the cost of adding extra ingredients. A customer would then be able to order dishes and pay for them, while the order would be sent through to the kitchen. This extension could see the project become a viable option for restaurants to use in practice. 

Another extension of this project would be to make it more customisable from the perspective of the restaurant owner in terms of diets. Diets are changing over time, and there are new dietary requirements that need to be catered for as time goes on. It would be useful if the owner of the restaurant could add more than just new allergens to the system. For example, Halal and Kosher are currently types of preparation method, but it would be useful if the owner could make more of these. One example would be deep fried foods, which are causing concern because of their high calorific content, high concentration of trans fats and the increase they cause in the risk of developing heart disease, diabetes and obesity \cite{deep_fried_mcdonnell_2023}. Therefore the restaurant owner might think that it would be useful if customers were able to filter out dishes that have been deep fried at some point. Currently, they would have to contact a software developer who would have to hard code this change into the system. It would be useful if this was not the case.


% Local Variables:
% mode: latex
% TeX-master: "report"
% End:
