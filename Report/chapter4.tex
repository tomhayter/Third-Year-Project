\chapter{Implementation}

\section{Deliverables}

\subsection{Ontology}

The final ontology (stored in \textit{`Menu.owl'}) contains the Honest Burgers \cite{honest_burgers_2023} menu. Each dish on the menu is a \textbf{Dish} in the ontology.

All the ingredients in a dish are also in the ontology. Each ingredient stores the number of calories that ingredient contains (on a class level), and the allergens it contains. There were times when I had to decide what level an ingredient was, for example the \textbf{PestoIngredient} which contains multiple ingredients in itself, but is listed as an ingredient in the dish. I thought it was not useful to a customer to list every single thing that makes up pesto, but just to make sure the allergens are stored. Otherwise the list of ingredients would be much larger.

What ingredients to put in a component is also a subjective task, and for this menu there is no intuitive set of components to make. I decided in this case to store the patty, burger bun, and fillings for each dish as the components.

The object properties in the ontology are \textit{hasPart}, which has the subProperties \textit{hasNutrient}, \textit{hasIngredient}, and \textit{hasComponent}. These are used to link a \textbf{Nutrient} to an \textbf{Ingredient}, an \textbf{Ingredient} to a \textbf{Component} and a \textbf{Component} to a \textbf{Dish} respectively. These components are transitive, because if a \textbf{Ingredient} is part of a \textbf{Component}, then it is also a part of any \textbf{Dish} that \textbf{Component} is a part of. The object property \textit{hasPreparationMethod} links a \textbf{PreparationMethod} such as Halal or Kosher to a \textbf{Dish}, to store which dishes have been prepared in that way.

The data properties in the ontology are \textit{hasCalories}, which links an \textbf{Ingredient} to an integer value of how many calories it contains. The data property \textit{hasGlutenFreeOption} links a \textbf{Dish} to a boolean value of whether there is a gluten free alternative of the dish available. This was an addition made after Study 1 (see Section \ref{section:Study1Results}).

\subsection{Restaurant User Interface}

The user interface for the restaurant owner opens onto the Main Menu Page. The user is presented with 4 button to navigate the system, to add to the menu, remove from the menu, search/edit the menu and quit the application. The quit button will save the ontology and close the application.

\begin{figure}[h]
    \centering
    \captionsetup{justification=centering}
    \includegraphics[width=0.50\textwidth]{screenshots/restaurantUI/AddIngredient.png}
    \caption{Restaurant UI - Add Ingredient Page}
    \label{fig:restaurantUI_add_ing_page}
\end{figure}

The Add section contains options to add either an allergen, ingredient, component or dish to the ontology. The page for adding an ingredient can be seen in Figure \ref{fig:restaurantUI_add_ing_page}. After attempting to add to the ontology, one of the message boxes in Figure A will be shown in response, depending on whether the attempted addition to the ontology is valid or not.

The remove section is very similar to the add section, except they show lists for a user to select an element to remove. Similarly, a message feedback box is shown to inform the user of the changes to the ontology.

There is also a section for searching and editing ingredients, components and dishes. This section was also inspired by the feedback from Study 1 (see Section \ref{section:Study1Results}). In the search page for a component, a user can scroll through all existing component and select one to edit.

\begin{figure}[h]
    \centering
    \captionsetup{justification=centering}
    \includegraphics[width=0.50\textwidth]{screenshots/restaurantUI/EditDish.png}
    \caption{Restaurant UI - Edit Dish Page}
    \label{fig:restaurantUI_edit_dish}
\end{figure}

Upon selection, the user is presented with a page that allows them to edit their selection. Figure \ref{fig:restaurantUI_edit_dish} shows the Edit Dish Page, which allows them to select new components for their dish, remove existing ones and change some properties about the dish.

\subsection{Customer User Interface}

The user interface for the customer also opens onto a main menu page where the user has 3 options. The user can search the menu, open the settings page or quit the application.

The settings page contains one option, the option to hide calorific information. This option will disable the ability to filter by calories, and also hide how many calories are in a dish and a dish with the ingredients modified.

\begin{figure}[h]
    \centering
    \captionsetup{justification=centering}
    \includegraphics[width=0.50\textwidth]{screenshots/customerUI/FilteredSearch.png}
    \caption{Customer UI - Search Page}
    \label{fig:customerUI_search}
\end{figure}

The search page (see Figure \ref{fig:customerUI_search}) contains the results of a search on the left hand side. Clicking on a dish will load the information for that dish. On the right hand side of the page is the list of filters. The first filter returns dishes suitable for various diets (Vegetarian, Halal). The second returns dishes with a gluten free alternative available. The third filter returns dishes with less than the maximum amount of calories specified. The fourth filter removes dishes that contain specified allergens.

Clicking on a dish in the search page will take you to page specific to that dish. This page displays the full list of ingredients in the dish on the left, along with the dietary, calorific and allergen information for that dish on the right. It also contains a link to add or remove ingredients from the dish and see updated information.

\begin{figure}[h]
    \centering
    \captionsetup{justification=centering}
    \includegraphics[width=0.50\textwidth]{screenshots/customerUI/Extras.png}
    \caption{Customer UI - Extras Page}
    \label{fig:customerUI_extras}
\end{figure}

The extras page (see Figure \ref{fig:customerUI_extras}) also contains the list of ingredients in the top left. However, this time clicking on one of the ingredients will remove it from the dish. There is a list of all ingredients in the bottom left of the page, and clicking on one of them will add it to the dish. On the right side of the page is a reset and calculate button. Clicking calculate will update the allergen and calorie information (also on the right hand side of the page). Because preparation is done on a dish level and not on an ingredient level, it is impossible to know if a dish will be Halal or Kosher purely based on the ingredients - so this information could not be included on this page.

\subsection{OntologyManager}


\section{Design Choices}
There were many things that did not go to plan, or that had to be changed along the way. Study 1 (see Section \ref{section:Study1Results}) brought up problems I had not considered.

Initially, I developed each user interface entirely in one class, with methods that would create and return a JFrame for each of the different pages. I would then change between JFrames by closing the old one and opening the new one. After making a few pages I soon realised this was a bad idea. I ended up making 19 different pages, and this would create a very convoluted class. After doing some research \cite{card_layout_2011}, I found that a better way to approach this was to make each page its own class, extending the JPanel class. Then, create a JPanel with the CardLayout layout that can switch between each page. CardLayout makes it so that each JComponent is added to a JPanel with a String that can be used to identify it, and then you can switch to the JPanel you desire by using its identifier. This improvement was worth the time it took to learn about, because it made my code much easier to navigate and therefore saved me time in the long run.

Another design choice that was made after my pilot study was to change the way I showed how things in a list were selected. Previously, users had a list of everything provided, and had to click on an element to select it. If a user wanted to select multiple elements in the list, they would have to hold the Control (Ctrl) key on their keyboard to do so. If Ctrl was not held, any previously selected elements would be unselected. This is similar to how file selection works in File Explorer, and it is the default selection model for a JList. However, as many of lists were long, it was not obvious that the previously selected elements were now unselected, as the user would have to scroll to check. My first solution was to change the selection model so that you only needed to click on each element to select it (no Ctrl required), and the only way to unselect an element was to click on it again. However, with such long lists of ingredients and components it would still not be clear what had been selected, and you would have to scroll through the entire list to check. This solution would therefore not comply with the reduction in short-term memory load principle that I set out to follow. Therefore, I ultimately decided on a different method where there is one list which contains all possible selections (List 1) and another list which contains all the selected elements (List 2). To select an element, the user would click on it in List 1. This will add it to List 2. To unselect an element, the user would click on it in List 2, and it will be removed from List 2. This makes it much more clear what has been selected as you can see all the elements in List 2 without seeing everything that has not been selected. It also supports the ability to select an element multiple times, which could be desirable in the case where a customer wants to add an ingredient twice to a dish. An example of this method can be seen in Figure \ref{fig:customerUI_extras}, where the list in the top left is List 2 and the list in the bottom left is List 1. A user can constantly see what is in their modified dish, which makes this task much easier.

One final point that was raised in the studies was that the system should include information regarding whether dishes have a gluten free alternative. This was a simple addition to make, I added a boolean data property for a dish called \textit{hasGlutenFreeOption}. Then a tickbox can be used to enter the information while creating a dish.

% Local Variables:
% mode: latex
% TeX-master: "report"
% End:
