\chapter{Results}

\section{Design Choices}
Initially, I started creating the user interface entirely in one class, with a method that would create and return a JFrame for each of the different pages. I would then change Frame by closing the old one and opening the new one. After making a few pages I quickly realised this was a bad idea. I ended up making 19 different pages, and this would create one very difficult to read and convoluted class. After doing some research\cite{cardLayout}, I found the better way to approach this was to make each page it's own class which extends JPanel, and then use a CardLayout to switch between each page. The CardLayout gives takes each page with a String that can be used to identify it (I made this a class variable of the page), and then you can show the panel you desire by using it's identifier. Despite the time it took to learn about, it made my code much more navigable and saved me time in the long run.

One design choice that was made after my pilot study was to change the way I showed how things in a list were selected. Previously, users had a list of everything provided, and had to click on an element to select it. If a user wanted to select multiple elements in the list, they would have to hold the Ctrl button to do so. If Ctrl was not held, any previously selected elements would be unselected. This is similar to how file selection works in File Explorer, and the default selection model for a JList. However, as many of lists were long and had to be scrolled, it wasn't obvious that the previously selected elements were now unselected. My first thought was to  solely change the selection model so that you only needed to click on each element to select it, and the only way to unselect an element was to select it again. However I ultimately decided on a different method, where there is one box to select elements from, and another box which contains all the selected elements. To unselect an element, click on it in the second list and it will automatically update. This makes it much more clear what has been selected as you can see them without seeing everything that hasn't been selected. It also supports the ability to select an element multiple times, which could be desirable in the case where a customer wants to add an ingredient twice to a dish.

One big consideration I had at the end was the purpose of the Component. With the particular menu I have chosen, components aren't particularly necessary. A burger is a whole meal and it doesn't consist of definitive sections, just ingredients. This caused some confusion at what components to make during the user study when the user was given free reign to make a dish.  Other menus lend themselves more to the idea of a Component however. If you have pie with mash potatoes and roasted vegetables - you have three very clear components to the final dish made up of many ingredients themselves. It lends itself to larger dishes that contain multiple parts, rather than menus where you order lots of small bits or somethiing that could be a Component itself. This is the reason I decided to keep Components in, as it gives the system flexibility to be used in different kinds of restaurants.

% Local Variables:
% mode: latex
% TeX-master: "report"
% End:
