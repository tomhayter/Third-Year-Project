\chapter{User Study 2 Results}
\label{app:study2results}


\begin{table}[h]
    \centering
    \begin{tabular}{ |p{0.5\textwidth}|c|c|c| }
    \hline
    \textbf{Task} & \textbf{Participant 1} & \textbf{Participant 2} & \textbf{Participant 3} \\
    \hline
    Add a Dairy Ingredient ``Parmesan'' to the menu, with 76 calories. Contains allergens: Milk. & \Checkmark & \Checkmark & \Checkmark \\
    \hline
    Add Component ``ChickenParmesan'' to system. Contains ingredients: ChickenBreast, Parmesan and Tomato. & \Checkmark & \Checkmark & \Checkmark \\
    \hline
    Add Dish ``ChickenParmesanBurger'' to system. Contains components: BurgerBun, Chicken Parmesan. & \Checkmark & \Checkmark & \Checkmark \\
    \hline
    Search for the ``ChickenParmesanBurger'' Dish. & \Checkmark & \Checkmark & \Checkmark \\
    \hline
    Remove the ``ChickenParmesanBurger'' Dish. & \Checkmark & \Checkmark & \Checkmark \\
    \hline
    Remove the ``ChickenParmesan'' Component. & \Checkmark & \Checkmark & \Checkmark \\
    \hline
    Remove the ``Parmesan'' Ingredient. & \Checkmark & \Checkmark & \Checkmark \\
    \hline
    Edit the calories in the BeefPatty ingredient from 451 to 551. & \Checkmark & \Checkmark & \Checkmark \\
    \hline
    Edit the name of the ``Cheddar'' ingredient to ``CheddarCheese''. & \Checkmark & \Checkmark & \Checkmark \\
    \hline
    Edit BeefBurgerFilling component to also include CheddarCheese. & \Checkmark & \Checkmark & \Checkmark \\
    \hline
    Edit the BeefBurger dish to state that it is Halal. & \Checkmark & \Checkmark & \Checkmark \\
    \hline
    Add new allergen ``Mustard''. & \Checkmark & \Checkmark & \Checkmark \\
    \hline
    Create a new dish: Steak Roll. Ingredients: BurgerBun, Steak, English Mustard (contains ``Mustard'' allergen), Red Onion and Rocket. & \Checkmark & \Checkmark & \Checkmark \\
    \hline
    Remove the Steak Roll dish. & \Checkmark & \Checkmark & \Checkmark \\
    \hline
    Attempt to add an ingredient with an empty name. & \Checkmark & \Checkmark & \Checkmark \\
    \hline
    Attempt to add a component with no ingredients. & \Checkmark & \Checkmark & \Checkmark \\
    \hline
    \end{tabular}
    \caption{Study 2 - Task results for restaurant owners}
    \label{tab:Study2ResultsRO}
\end{table}


\begin{table}[h]
    \centering
    \begin{tabular}{ |p{0.5\textwidth}|c|c|c| }
    \hline
    \textbf{Task} & \textbf{Participant 1} & \textbf{Participant 2} & \textbf{Participant 3} \\
    \hline
    Find how many calories are in a BeefBurger. & \Checkmark & \Checkmark & \Checkmark \\
    \hline
    Find all dishes that contain less than 600 calories. & \Checkmark & \Checkmark & \Checkmark \\
    \hline
    Find all dishes suitable for vegetarians. & \Checkmark & \Checkmark & \Checkmark \\
    \hline
    Find all dishes that are Halal. & \Checkmark & \Checkmark & \Checkmark \\
    \hline
    Find all dishes that have a gluten free alternative. & \Checkmark & \Checkmark & \Checkmark \\
    \hline
    Find all dishes that contain no egg or milk. & \Checkmark & \Checkmark & \Checkmark \\
    \hline
    Find dishes that are Halal, suitable for vegetarians, contain no egg and are under 700 calories. & \Checkmark & \Checkmark & \Checkmark \\
    \hline
    Find how many calories are in a PestoBurger with added SmokedBacon and Cheddar, but with the Tomato removed. & \Checkmark & \Checkmark & \Checkmark \\
    \hline
    Find which allergens are contained in the TributeBurger. & \Checkmark & \Checkmark & \Checkmark \\
    \hline
    Search with the calories turned off. & \Checkmark & \Checkmark & \Checkmark \\
    \hline
    Find which allergens are contained in a BeefBurger with the BeefPatty removed. & \Checkmark & \Checkmark & \Checkmark \\
    \hline
    \end{tabular}
    \caption{Study 2 - Task results for customers}
    \label{tab:Study2ResultsC}
\end{table}


\begin{table}[h]
    \centering
    \captionsetup{justification=centering}
    \begin{tabular}{ |p{0.25\textwidth}|p{0.25\textwidth}|p{0.25\textwidth}|p{0.25\textwidth}| }
    \hline
    \textbf{Question} & \textbf{Participant 1} & \textbf{Participant 2} & \textbf{Participant 3} \\
    \hline
    How fast was the system to respond? (1-5) & 5 & 4 & 5 \\
    \hline
    How intuitive was the system? (1-5) & 5 & 5 & 5 \\
    \hline
    How easy was the system to use? (1-5) & 5 & 5 & 5 \\
    \hline
    How easy was it to pick up after not using the system for a month? (1-5) & 5 & 5 & 5 \\
    \hline
    Was there anything you wish was explained to you better before use? & No & No & No \\
    \hline
    What did you like about the system? & Ability to search for existing entities, and edit entities directly. & Logical, intuitive, calculates information for you. Easy to use and obvious how to use interface. & Laid out logically and clearly, edit directly from search was useful. Fast and responsive. \\
    \hline
    Where do you think the system could be improved? & Components seem redundant, could they be removed? Add prices. & Not clear what components should be. Text search for lists to find item. & Ability to delete from edit page. \\
    \hline
    Has the system improved since the previous study? & Yes & Yes & Yes \\
    \hline
    Did the system offer informative feedback to your actions? & Yes & Yes & Yes \\
    \hline
    Was the system consistent throughout? & Yes & Yes & Yes \\
    \hline
    Did you feel in control of the system? & Yes & Yes & Yes \\
    \hline
    Did you feel the need to remember lots of information? & No & No & No \\
    \hline
    Did you feel that you could easily reverse changes you made to the system? & Yes & Yes & Yes \\
    \hline
    General Feedback. & Logical process, easy to visualise. All pages look similar, hard to distinguish pages other than title. & Multiple selection is more clear. Beef is not an allergen, could be misleading. & Editing is useful, multiple selection has improved. Clearly laid out, easy to use and intuitive. \\
    \hline
    \end{tabular}
    \captionsetup{justification=centering}
    \caption[Study 2 - Question responses from restaurant owners]{Study 2 - Question responses from restaurnat owners. Questions marked ``(1-5)'' were marked on a Likert scale.}
    \label{tab:Study2AnswersRO}
\end{table}

\begin{table}[h]
    \centering
    \captionsetup{justification=centering}
    \begin{tabular}{ |p{0.25\textwidth}|p{0.25\textwidth}|p{0.25\textwidth}|p{0.25\textwidth}| }
    \hline
    \textbf{Question} & \textbf{Participant 1} & \textbf{Participant 2} & \textbf{Participant 3} \\
    \hline
    How fast was the system to respond? (1-5) & 5 & 5 & 5 \\
    \hline
    How intuitive was the system? (1-5) & 5 & 5 & 4 \\
    \hline
    How easy was the system to use? (1-5) & 5 & 5 & 5 \\
    \hline
    Was there anything you wish was explained to you better before use? & No. & No. & No. \\
    \hline
    What did you like about the system? & Very easy to use, extensive dietary options. Important to be able to turn off calories. & Very quick, customers don't want to be waiting. Multiple selection has improved. Contains everything necessary and nothing more. & New gluten free option. Search by calories so you don't find out after choosing dish. Allergens were clearly presented, felt in control as someone who looks for that information. \\
    \hline
    Where do you think the system could be improved? & Forgot about settings, move them to search page? Sentence to explain whether filter includes or excludes selection. & Calorie slider can't click to move, you have to drag. Could view specific gluten free alternative ingredients etc. & Put settings in search page, forgot they were there until asked. \\
    \hline
    Did the system offer informative feedback to your actions? & Yes & Yes & Yes \\
    \hline
    Was the system consistent throughout? & Yes & Yes & Yes \\
    \hline
    Did you feel in control of the system? & Yes & Yes & Yes \\
    \hline
    Did you feel the need to remember lots of information? & No. & No. & No. \\
    \hline
    Did you feel that you could easily reverse changes you made to the system? & Yes & Yes & Yes \\
    \hline
    General Feedback. & None given. & Very intuitive, covered all bases. & Easy to use and flexible. \\
    \hline
    \end{tabular}
    \captionsetup{justification=centering}
    \caption[Study 2 - Question responses from customers ]{Study 2 - Question responses from customers. Questions marked ``(1-5)'' were marked on a Likert scale.}
    \label{tab:Study2AnswersC}
\end{table}

% Local Variables:
% mode: latex
% TeX-master: "report"
% End:
