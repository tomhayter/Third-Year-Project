\chapter{Results}

In this chapter I will discuss the results of the second user study that I conducted during my project.

\section{Study 2}
\label{section:Study2Results}

Study 2 was carried out after I finished development on the system. The aim of this study was to evaluate how well the system met the requirements I set out in Section \ref{section:SystemRequirements}, and evaluate whether it would be able to be used in the real world. I also carried out Study 2 with 6 participants. The 3 participants who acted as a restaurant owner in Study 1 played the same role in this study. I asked them to carry out the same tasks, along with a couple of new tasks concerning new functionality that had been introduced since Study 1. All the same questions from Study 1 were asked, along with one new one for the restaurant owners to evaluate how well they could use the system without any training one month later. On top of this, I also asked some questions to evaluate how well the system fit \textit{The eight golden rules of interface design} \cite{shneiderman}. The full list of tasks and questions can be seen in Appendix \ref{app:study2}, while the results can be found in Appendix \ref{app:study2results}.

\subsection{Customer UI Study}

The results of the customer study demonstrated that every participant was able to complete every task without any training or guidance. They reported that the system was very quick and easy to use. The calorie slider received praise because it removes the disappointment of finding a dish you like only to see it has more calories than you would like. The importance of being able to turn off calories was also reiterated by the participants. They stated that all the information was clearly displayed and easy to find. Two of the three participants said that because they spent so much time in the search page, they almost forgot about the main menu and settings page.

This time, the Customer UI was rated, on average, 5/5 for speed, 4.6/5 for intuitiveness and 5/5 for ease of use. It also complied to all four of the applicable \textit{golden rules of user interface} according to each participant. 

\subsection{Restaurant UI Study}

The restaurant owner study also showed that every participant was able to complete every task set for them. They noted that the ability to search for ingredients, as well as edit them directly made them feel much more in control of the system, and made it easier to fix mistakes. They also noted that every part of the system was laid out in a clear and logical way, which made it obvious how to do what a restaurant owner would want to do. Conversely, they also noted that it is not clear what a Component should be in the context of this menu. One participant also noted that it would be useful to be able to delete a component from the edit page. They also appreciated the new method for selecting multiple items, which was much more intuitive. Overall, they all thought it had improved since the system in Study 1.

The Restaurant UI was rated, on average, 4.6/5 for speed, 5/5 for intuitiveness and 5/5 for ease of use. 5/5 was also the average score for how easy it was to pick up after not using the system for a month. On top of this, each participant stated that it complied to all five of the applicable \textit{golden rules of user interface}.

\subsection{Interpreting the Results}

As this study was conducted after development had finished on my system, the results will form part of the evaluation of the system, which can be seen in Section \ref{section:Evaluation}.

% Local Variables:
% mode: latex
% TeX-master: "report"
% End:
