\documentclass{article}
\usepackage{graphicx}

\begin{document}
\begin{titlepage}
\begin{center}
\vspace*{5cm}

\Huge
\textbf{An ontology-based restaurant system}

\vspace{1cm}

\Large
\textbf{Thomas Hayter}

\vspace{1cm}

\today
\end{center}
\end{titlepage}

\tableofcontents

\section{Intro}
I used this \cite{pizza_tutorial_pdf}.

\section{Requirements of the ontology}

These are the requirements of the ontology so that it can store all the information that I would like it to:

\begin{itemize}
\item Ingredient list for every meal on a menu, split into the components of the dish.
\item Calorific content of each meal.
\item Allergen information for each meal.
\item Types of customer e.g. Vegetarians and those with Coelaic disease.
\end{itemize}

These are possible additions that could be made to the ontology, but are not necessary for the MVP:

\begin{itemize}
\item The ability to query a dish with ingredients added or removed.
\item The ability to query a dish based on how a dish has been cooked, e.g. Which meals have not used a deep fryer to be made?
\end{itemize}

The requirements of the user interface for restaurant owner:

\begin{itemize}
\item Add and remove meals using ingredients and components list.
\item Add and remove ingredients and components, in same page as above and separate.
\item Add and remove customer types (dietry requirements and allergens).
\item Query the ontology.
\end{itemize}

Dietary Requirements to filter by:

\begin{itemize}
\item Vegetarian
\item Pescetarian
\item Vegan
\item
\end{itemize}

Allergies\cite{burks2001food}:

\begin{itemize}
\item Peanut
\item Ceoliac (gluten)
\item Wheat
\item Cow's milk
\item Eggs
\item Fish
\item Shellfish
\item Tree nuts
\item Soybeans
\end{itemize}


\section{Bibliography}
\bibliographystyle{unsrt}
\bibliography{refs}

\end{document}
